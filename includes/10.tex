\section{Überblick}

\begin{frame}
  {Nebensätze und unabhängige Sätze}
  \begin{itemize}[<+->]
    \item \citet{Schaefer2018b}
  \end{itemize}
\end{frame}


\section{Relativsätze}

\begin{frame}
  {Relativsätze als etwas andere VL-Sätze}
  \pause
  Das \tuerkis{Relativelement} wird nach links gestellt. Das \alert{Verb} bleibt rechts.\\
  \pause
  \begin{center}
    \adjustbox{max width=0.4\textwidth}{%
      \begin{minipage}{0.7\textwidth}
      \begin{forest}
        [NP, calign=child, calign child=2
          [Art, tier=preterminal
            [\it einen]
          ]
          [\bf N, tier=preterminal
            [\it Tofu]
          ]
          [RS, calign=first
            [NP\Sub{1}, tier=preterminal, tuerkistree
              [\it der, narroof, name=BeweDer]
            ]
            [VP, calign=last
              [NP, tier=preterminal, baseline
                [\it mir, narroof]
              ]
              [\Ti, tier=preterminal, tuerkistree]
              {\draw[dotted, tuerkis, thick, ->] (.south) |- ++(0,-4.5em) -| (BeweDer.south);}
              [Ptkl, tier=preterminal
                [nicht]
              ]
              [\bf V, calign=last
                [\bf V, tier=preterminal
                  [\it geschmeckt]
                ]
                [\bf V, tier=preterminal, bluetree
                  [\it hat]
                ]
              ]
            ]
          ]
        ]
      \end{forest}
    \end{minipage}
    }
    \pause
    \hspace{0.1\textwidth}\adjustbox{max width=0.35\textwidth}{%
      \begin{minipage}{0.35\textwidth}
      \begin{forest}
        [RS, Ephr
          [XP\UpSub{relativ}{1}, Eobl, baseline, tuerkis]
          [VP\\{[\ldots\tuerkis{\Ti}\ldots]}, Eobl]
        ]
      \end{forest}
      \end{minipage}
    }
  \end{center}
  \pause
  \Halbzeile
  \begin{itemize}[<+->]
    \item Relativelement
      \begin{itemize}[<+->]
        \item \alert{Bedeutung}: Bezugs-Substantiv
        \item \alert{Genus, Numerus}: Kongruenz mit Bezugs-Substantiv
        \item \alert{Kasus\slash PP-Form}: gemäß Status als Ergänzung\slash Angabe im RS
      \end{itemize}
  \end{itemize}
\end{frame}


\begin{frame}
  {Komplexe Einbettung des Relativelements}
  \pause
  Das \tuerkis{Relativelement} als pränominaler Genitiv nimmt die Matrix-NP mit.\\
  \pause
  \Halbzeile
  \centering
  \begin{forest}
    [NP, calign=child, calign child=2
      [Art, tier=preterminal
        [\it der]
      ]
      [\bf N, tier=preterminal
        [\it Tofu]
        {\draw [->, bend right=30] (.south) to node [below, near start] {\footnotesize\textsc{Genus,Numerus}} (RekDessen.south);}
      ]
      [RS, calign=first
        [NP\Sub{1}, calign=first, tuerkistree
          [NP, tier=preterminal
            [\it dessen, narroof, name=RekDessen]
          ]
          [\bf N, tier=preterminal
            [\it Geschmack, name=RekGeschmack]
            {\draw [->, bend left=25] (.south) to node [below, near start] {\footnotesize\textsc{Kasus}} (RekDessen.south);}
          ]
        ]
        [VP, calign=last
          [NP, tier=preterminal
            [\it ich, narroof]
          ]
          [\Ti, tuerkistree]
          [\bf V, tier=preterminal
            [\it mag]
            {\draw [->, bend left=15] (.south) to node [below, near start] {\footnotesize\textsc{Kasus}} (RekGeschmack.south);}
          ]
        ]
      ]
    ]
  \end{forest}
\end{frame}



\subsection{Objektsätze}

\begin{frame}
  {Objektsätze}
  \pause
  \begin{exe}
    \ex{\label{ex:komplementsaetze127} Michelle weiß, [\rot{dass} die Corvette nicht anspringen wird].}
    \pause
    \ex\label{ex:komplementsaetze128}
    \begin{xlist}
      \ex{\label{ex:komplementsaetze129} Michelle will wissen, [\rot{wer} die Corvette gewartet hat].}
      \pause
      \ex{\label{ex:komplementsaetze130} Michelle will wissen, [\rot{ob} die Corvette gewartet wurde].}
    \end{xlist}
  \end{exe}
  \pause
  \Halbzeile
  \alert{Achtung: \textit{ob} ist eigentlich nur ein w-Wort ohne w (vgl.\ engl.\ \textit{whether}).}\\
  \pause
  \Halbzeile
\end{frame}

\begin{frame}
  {Regierende Verben und Alternationen}
  \pause
  \alert{Drei primäre Muster}, welche Satz-Objekte Verben regieren.\\
  \pause\Halbzeile
  \begin{exe}
    \ex\label{ex:komplementsaetze131}
    \begin{xlist}
      \ex[]{\label{ex:komplementsaetze132} Michelle behauptet, \alert{dass} die Corvette nicht anspringt.}
      \pause
      \ex[*]{\label{ex:komplementsaetze133} Michelle behauptet, \rot{wie\slash ob} die Corvette nicht anspringt.}
    \end{xlist}
    \pause
    \ex\label{ex:komplementsaetze134}
    \begin{xlist}
      \ex[*]{\label{ex:komplementsaetze135} Michelle untersucht, \rot{dass} der Vergaser funktioniert.}
      \pause
      \ex[]{\label{ex:komplementsaetze136} Michelle untersucht, \alert{wie\slash ob} der Vergaser funktioniert.}
    \end{xlist}
    \pause
    \ex\label{ex:komplementsaetze137}
    \begin{xlist}
      \ex[]{\label{ex:komplementsaetze138} Michelle hört, \alert{dass} die Nockenwelle läuft.}
      \pause
      \ex[]{\label{ex:komplementsaetze139} Michelle hört, \alert{wie\slash ob} die Nockenwelle läuft.}
    \end{xlist}
  \end{exe}
  \pause\Halbzeile
  Außerdem: \textit{dass} alterniert oft mit \textit{zu}-Infinitiv.\\
  \pause
  \Halbzeile
  \begin{exe}
  \ex\label{ex:komplementsaetze140}
  \begin{xlist}
    \ex{\label{ex:komplementsaetze141} Michelle glaubt, [\alert{dass} sie das Geräusch erkennt].}
    \pause
    \ex{\label{ex:komplementsaetze142} Michelle glaubt, [das Geräusch \alert{zu} erkennen].}
  \end{xlist}
  \end{exe}
\end{frame}

\begin{frame}
  {Stellung von Adverbial- und Komplementsätzen}
  \pause
  \Halbzeile
  \begin{exe}
  \ex\label{ex:komplementsaetze146}
  \begin{xlist}
    \ex[]{\label{ex:komplementsaetze147} \alert{[Dass sie unseren Kuchen mag]}, hat Sarah uns eröffnet.}
    \pause
    \ex[]{\label{ex:komplementsaetze148} Sarah hat uns eröffnet, \alert{[dass sie unseren Kuchen mag]}.}
    \pause
    \ex[?]{\label{ex:komplementsaetze149} Sarah hat uns, \rot{[dass sie unseren Kuchen mag]}, eröffnet.}
  \end{xlist}
    \pause

  \ex\label{ex:komplementsaetze150}
  \begin{xlist}
    \ex[]{\label{ex:komplementsaetze151} \alert{[Ob Pavel unseren Kuchen mag]}, haben wir uns oft gefragt.}
    \pause
    \ex[]{\label{ex:komplementsaetze152} Wir haben uns oft gefragt, \alert{[ob Pavel unseren Kuchen mag]}.}
    \pause
    \ex[?]{\label{ex:komplementsaetze153} Wir haben uns, \rot{[ob Pavel unseren Kuchen mag]}, oft gefragt.}
  \end{xlist}
    \pause
  \ex\label{ex:komplementsaetze154}
  \begin{xlist}
    \ex[]{\label{ex:komplementsaetze155} \alert{[Wer die Rosinen geklaut hat]}, wollen wir endlich wissen.}
    \pause
    \ex[]{\label{ex:komplementsaetze156} Wir wollen endlich wissen, \alert{[wer die Rosinen geklaut hat]}.}
    \pause
    \ex[?]{\label{ex:komplementsaetze157} Wir wollen, \rot{[wer die Rosinen geklaut hat]}, endlich wissen.}
  \end{xlist}
  \end{exe}
  \pause
  \begin{itemize}[<+->]
    \item Fast immer Bewegung nach links oder Rechtsversetzung \alert{hinter VK}!\\
    \item \grau{Fehlendes Schema für Rechtsversetzung: Transferaufgabe im Buch.}
  \end{itemize}
  \pause
\end{frame}


\begin{frame}
  {Korrelate}
\end{frame}

\section{Das Feldermodell}

\section{Vorschau}

