
\section{Überblick}

\begin{frame}
  {Andere Phrasentypen}
  \onslide<+->
  \begin{itemize}[<+->]
    \item Adjektivphrasen
    \item Präpositionalphrasen
    \item Adverbphrasen
    \item Koordination
    \item Komplementiererphrase
      \Zeile
    \item \citet[12.2,12.4--12.7]{Schaefer2018b}
  \end{itemize}
\end{frame}

\section{Adjektivphrasen}

\section{Präpositionalphrasen}

\section{Adverbphrasen}

\section{Koordination}

\section{Komplementiererphrase}

\begin{frame}
  {Komplementiererphrasen = eingeleitete Nebensätze}
  \pause
  \begin{exe}
    \ex\label{ex:komplementiererphrase111}
    \begin{xlist}
      \ex[]{\label{ex:komplementiererphrase112} Der Arzt möchte, [dass [der Privatpatient die Rechnung \alert{bezahlt}]].}
      \pause
      \ex[*]{\label{ex:komplementiererphrase113} Der Arzt möchte, [dass [der Privatpatient \rot{bezahlt} die Rechnung]].}
      \pause
      \ex[*]{\label{ex:komplementiererphrase114} Der Arzt möchte, [dass [\rot{bezahlt} der Privatpatient die Rechnung]].}
    \end{xlist}
  \end{exe}
  \pause
  \Halbzeile
  \centering
  \begin{forest}
    [KP, calign=first
      [\bf K, tier=preterminal
        [\it dass, name=Kpkopf]
      ]
      [\alert{VP}, tier=preterminal
        [\it der Kassenpatient \alert{geht}, narroof]
      ]
    ]
  \end{forest}\\
  \pause
  \Zeile
  \alert{Verb-Letzt-Stellung!}\\
\end{frame}



\begin{frame}
  {Komplementiererphrase | Schema}
  \centering
  \begin{forest}
    phrasenschema
    [KP, Ephr, calign=first
      [K, Ehd, name=Kpkopf]
      [VP, Eobl]
      {\draw [bend left=45, <-] (.south) to (Kpkopf.south);}
    ]
  \end{forest}
\end{frame}


\section{Vorschau}

