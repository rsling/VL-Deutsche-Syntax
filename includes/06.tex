

\section{Überblick}

\begin{frame}
  {Überblick: Konstituenten und Phrasen}
  \pause
  \begin{itemize}[<+->]
    \item Phrasen und Köpfe
    \item Strukur der deutschen \alert{Nominalphrase}
    \item (regierte) Attribute
      \Zeile
    \item \citet[Abschnitt~12.3]{Schaefer2018b}
  \end{itemize}
\end{frame}

\begin{frame}
  {Syntax und (bildungssprachliche) Funktion}
  \pause
  \begin{itemize}[<+->]
    \item \alert{hohe Komplexität} des syntaktischen Systems
    \item \rot{Regularitätensystem kaum vollständig explizit lernbar}
    \item überall \alert{starke Interaktion mit Semantik, Pragmatik usw.}
    \item \alert{Kompositionalität}
      \Zeile
    \item Der Versuch, Funktionen zu beschreiben, ohne Formsystem zu kennen,\\
      wäre in der Syntax völlig absurd.
      \Zeile
    \item reduzierte Syntax = erhebliche Einschränkung des Ausdrucks
    \item komplexe schriftsprachliche Syntax, ggf.\ \rot{Rezeptionsprobleme}
  \end{itemize}
\end{frame}


\section{Phrasentypen}

\begin{frame}
  {Jede Phrase hat genau einen Kopf}
  \pause
  \resizebox{\textwidth}{!}{
    \begin{tabular}{lll}
      \toprule
      \textbf{Kopf} & \textbf{Phrase} & \textbf{Beispiel} \\
      \midrule
      Nomen (Substantiv, Pronomen) & Nominalphrase (NP) & \textit{die tolle \alert{Auf"|führung}} \\
      Adjektiv & Adjektivphrase (AP) & \textit{sehr \alert{schön}} \\
      Präposition & Präpositionalphrase (PP) & \textit{\alert{in} der Uni} \\
      Adverb & Adverbphrase (AdvP) & \textit{total \alert{offensichtlich}} \\
      Verb & Verbphrase (VP) & \textit{Sarah den Kuchen gebacken \alert{hat}} \\
      Komplementierer & Komplementiererphrase (KP) & \textit{\alert{dass} es läuft} \\
      \bottomrule
    \end{tabular}
  } 
  \pause
  \Halbzeile
  \begin{itemize}[<+->]
    \item Der Kopf bestimmt den \alert{internen Aufbau} der Phrase.
    \item Der Kopf bestimmt die \alert{externen kategorialen Merkmale} der Phrase\\
      und so das syntaktische Verhalten der Phrase (Parallele: \alert{Kompositum}).
  \end{itemize}
\end{frame}


\begin{frame}
  {Wieviele Wortklassen? Wieviele Phrasentypen?}
  \pause
  \begin{itemize}[<+->]
    \item \alert{Phrasentyp: passend zur Wortklasse des Kopfes}
    \item maximal so viele Phrasentypen wie Wortklassen
    \item aber: nicht alle Wortklassen kopffähig (\alert{Funktionswörter})
      \Zeile
    \item heute nur der wahrscheinlich komplexeste nicht-satzförmige Phrasentyp:
      \begin{itemize}[<+->]
        \item Nominalphrase
      \end{itemize}
  \end{itemize}
\end{frame}

\section{Nominalphrasen}

\begin{frame}
  {Ziemlich volle NP-Struktur mit Substantiv-Kopf}
  \pause
  \centering
  \begin{forest}
    [NP, calign=child, calign child=3
      [Art
        [\it die]
      ]
      [AP
        [\it antiken, narroof]
      ]
      [\textbf{N}, tier=preterminal
        [\it Zahnbürsten]
      ]
      [NP, tier=preterminal
        [\it des Königs, narroof
        ]
      ]
      [RS
        [\it die nicht benutzt wurden, narroof]
      ]
    ]
  \end{forest}
  \pause
  \Zeile
  \begin{itemize}[<+->]
    \item \textit{die antiken Zahnbürsten}: \alert{Kongruenz}
    \item Baum über dem \alert{genusfesten} Kopf aufgebaut
    \item \alert{inneres Rechtsattribut} \textit{des Königs}
    \item \alert{Relativsatz} \textit{die nicht benutzt wurden}
  \end{itemize}
\end{frame}


\begin{frame}
  {Struktur mit pronominalem Kopf}
  \pause
  \centering
  \begin{forest}
    [NP, calign=child, calign child=1
      [\textbf{N}, tier=preterminal
        [\it einige]
      ]
      [NP, tier=preterminal
        [\it des Königs, narroof
        ]
      ]
      [RS
        [\it die geklaut wurden, narroof]
      ]
    ]
  \end{forest}
  \pause
  \Zeile
  \begin{itemize}[<+->]
    \item links vom Kopf: \rot{nichts}
    \item Determinierung erfolgt beim Pronomen \alert{im Kopf}.
    \item Determinierung schließt NP nach links ab.
    \item → \alert{Also kann links vom Pron-Kopf nichts stehen!}
  \end{itemize}
\end{frame}


\begin{frame}
  {Nominalphrase allgemein (Schema)}
  \pause
  \centering
  \begin{forest}
    phrasenschema
    [NP, Ephr
      [Art, Eopt, Emult, [NP\Sub{Genitiv}, Eopt]]
      [AP, Eopt, Erec]
      [N, Ehd, name=Nkopf]
      [innere Rechtsattribute, Eopt, Erec]
      {\draw [bend left=45, dashed,<-] (.south) to (Nkopf.south);}
      [RS, Eopt, Erec]
    ]
  \end{forest}
\end{frame}


\begin{frame}
  {Regierte Rechtsattribute}
  \pause
  \begin{exe}
    \ex die \gruen{Beachtung} \alert{[ihrer Lyrik]}
    \pause
    \ex mein \gruen{Wissen} \alert{[um die Bedeutung der komplexen Zahlen]}
    \pause
    \ex die \gruen{Überzeugung}, \alert{[dass die Quantenfeldtheorie \\
    die Welt korrekt beschreibt]}
    \pause
    \ex die \gruen{Frage}, \alert{[ob sich die Luftdruckanomalie von 2018 wiederholen wird]}
    \pause
    \ex die \gruen{Frage} \alert{[nach der möglichen Wiederholung der Luftdruckanomalie]}
  \end{exe}
  \pause
  \Halbzeile
  \begin{itemize}[<+->]
    \item typisch: postnominale Genitive, PPs, satzförmige Recta
  \end{itemize}
\end{frame}


\begin{frame}
  {Korrespondenzen zwischen Verben und Nomina(lisierungen)}
  \pause
  Viele Substantive entsprechen einem Verb mit bestimmten Rektionsanforderungen.\\
  \pause
  \Zeile
  \begin{exe}
    \ex\label{ex:rektionundvalenzindernp031}
    \begin{xlist}
      \ex{\label{ex:rektionundvalenzindernp032} \orongsch{Sarah} \alert{verziert} \gruen{[den Kuchen]}.}
      \pause
      \ex{\label{ex:rektionundvalenzindernp033} [Die \alert{Verzierung} \gruen{[des Kuchens]} \orongsch{[durch Sarah]}]}
      \pause
      \ex{\label{ex:rektionundvalenzindernp034} [Die \alert{Verzierung} \gruen{[von dem Kuchen]} \orongsch{[durch Sarah]}]}
    \end{xlist}
  \end{exe}
  \pause
  \Zeile
  \begin{itemize}[<+->]
    \item \gruen{Akkusativ} beim transitiven Verb $\Leftrightarrow$ \gruen{Genitiv}\slash\gruen{von-PP} beim Substantiv
    \item \orongsch{Nominativ} beim transitiven Verb $\Leftrightarrow$ \orongsch{durch-PP} beim Substantiv
    \item Beim nominalen Kopf: alle Ergänzungen optional
  \end{itemize}
\end{frame}


\begin{frame}
  {Alternative Korrespondenzen für Nominative}
  \pause
  \begin{exe}
    \ex\label{ex:rektionundvalenzindernp035}
    \begin{xlist}
      \ex{\label{ex:rektionundvalenzindernp036} \orongsch{[Sarah]} rettet [den Kuchen] [vor dem Anbrennen].}
      \pause
      \ex{\label{ex:rektionundvalenzindernp037} [\orongsch{[Sarahs]} Rettung [des Kuchens] [vor dem Anbrennen]]}
    \end{xlist}
  \end{exe}
  \pause
  \begin{itemize}[<+->]
    \item \orongsch{Nominativ} beim transitiven Verb $\Leftrightarrow$\\
      \orongsch{pränominaler Genitiv} beim Substantiv
  \end{itemize}
  \pause
  \Halbzeile
  \begin{exe}
    \ex[ ]{\gruen{[Die Schokolade]} wirkt gemütsaufhellend.}
    \pause
    \ex[ ]{[Die Wirkung \gruen{[der Schokolade]}] ist gemütsaufhellend.}
    \pause
    \ex[?]{[Die Wirkung \gruen{[von der Schokolade]}] ist gemütsaufhellend.}
    \pause
    \ex[*]{[\gruen{[Der Schokolade]} Wirkung] ist gemütsaufhellend.}
  \end{exe}
  \pause
  \begin{itemize}[<+->]
    \item \gruen{Nominativ} beim intransitiven Verb $\Leftrightarrow$\\
      \gruen{prä-\slash postnominaler Genitiv}\slash\gruen{von-PP} beim Substantiv
  \end{itemize}
\end{frame}


\begin{frame}
  {Komplexität der NP | Sätze und NPs}
  \onslide<+->
  \onslide<+->
  Die NP erreicht eine außergewöhnliche Komplexität,\\
  weil sich ganze Sätze als NP verpacken lassen.\\
  \onslide<+->
  \Zeile
  \begin{exe}
    \ex{ } \grau{Martinas Freundin ist wieder zuhause.}\\
      \rot{Martina} \alert{teilt} \gruen{ihr} \alert{mit}, \orongsch{dass die Pferde bereits gefüttert wurden}.
    \onslide<+->
    \Zeile
    \ex{ } [\rot{[Martinas]} \alert{Mitteilung} [\gruen{an ihre Freundin}, \grau{[die wieder zuhause ist]}],\\
      { }\orongsch{[dass die Pferde bereits gefüttert wurden]}],\\
      (kam gerade noch rechtzeitig.)
  \end{exe}
\end{frame}


\begin{frame}
  {Baum für die NP}
  \onslide<+->
  \onslide<+->
  \centering
  \begin{forest}
    [NP, calign=child, calign child=2, tier=root
      [NP, tier=subroot, rottree
        [\it Martinas, narroof, tier=terminal]
      ]
      [\textbf{N}, tier=subroot, bluetree
        [\it Mitteilung, tier=terminal]
      ]
      [PP, tier=subroot, gruen
        [P, calign=child, calign child=1, gruennode
          [\it an, tier=terminal, gruennode]
          [NP, calign=child, calign child=2, gruennode
            [Art, tier=preterminal, gruennode
              [\it ihre, tier=terminal, gruennode]
            ]
            [N, tier=preterminal, gruennode
              [\it Freundin, gruennode]
            ]
            [RS, tier=preterminal, grautree
              [\it die \ldots\ ist, narroof]
            ]
          ]
        ]
      ]
      [KP, tier=subroot, orongschtree
        [\it dass \ldots\ wurden, narroof, tier=terminal]
      ]
    ]
  \end{forest}

\end{frame}

\section{Vorschau}

\begin{frame}
  {Andere Phrasentypen}
  \onslide<+->
  \begin{itemize}[<+->]
    \item Adjektivphrasen
    \item Präpositionalphrasen
    \item Adverbphrasen
    \item Koordination
    \item Komplementiererphrase
      \Zeile
    \item \citet[12.2,12.4--12.7]{Schaefer2018b}
  \end{itemize}
\end{frame}
