
\section{Organisation}


\begin{frame}
  {Ablauf und Inhalte der Vorlesung}
  \begin{itemize}
    \item 13 Sitzungen über Grammatik des Deutschen und Linguistik
    \item drei Sitzungen: Anwendung auf Textebene, Übungen für die Klausur 
      \vspace{\baselineskip}
    \item Meine Inhalte entsprechen meiner \alert{\textit{Einführung in\\
      die grammatische Beschreibung des Deutschen:}}\\
      \rot{\textit{Dritte, überarbeitete und erweiterte Auflage}}
    \item \url{http://langsci-press.org/catalog/book/224} (\alert{open access})
      \vspace{\baselineskip}
    \item Bei Amazon für 20€: \url{https://www.amazon.de/dp/3961101183/}
  \end{itemize}
\end{frame}

\begin{frame}
  {Fragen und Interaktion}
  \begin{itemize}
    \item Interaktion in einer VL mit 900 Teilnehmer*innen ist ausgeschlossen.
      \vspace{\baselineskip}
    \item Wenn Sie Fragen zum Stoff oder zum Buch haben:
      \texttt{roland.schaefer@fu-berlin.de}
    \item Ich würde geeignete Fragen auch gerne in meinem Blog beantworten:\\
      \url{http://grammatick.de}
      \vspace{\baselineskip}
    \item \rot{Bitte beachten Sie folgende Hinweise zur Email-Kommunikation:\\
        \url{http://rolandschaefer.net/?page_id=1736}}
  \end{itemize}
\end{frame}

\begin{frame}
  {Der Plan für heute}
  \pause
  \begin{itemize}
    \item Grammatik
      \begin{itemize}
        \item Grammatik als System
        \item Kern und Peripherie des Systems
        \item Norm und Beschreibung, Regel und Regularität
      \end{itemize}
      \vspace{\baselineskip}
      \pause
    \item Grammatik in Schule und Studium
      \begin{itemize}
        \item Bildungssprache
        \item Sprachbetrachtung
        \item Welche Grammatik für das Germanistikstudium?
      \end{itemize}
      \vspace{\baselineskip}
      \pause
    \item EGBD3: Kapitel 1 bis 3
      \pause
    \item \alert{Sie müssen irgendwann vor der Klausur diese Kapitel durcharbeiten.}
  \end{itemize}
\end{frame}


\section{Grammatik}

\begin{frame}
  {Deutsche Sätze erkennen und interpretieren}
  \pause
  \begin{exe}
    \ex Dies ist ein Satz.
  \pause
    \ex Satz dies ein ist.
  \pause
    \ex Kno kna knu.
  \pause
    \ex This is a sentence.
  \pause
    \vspace{\baselineskip}
    \ex Dies ist ein Satz
  \end{exe}
\end{frame}


\begin{frame}
  {Form und Bedeutung: Kompositionalität}
  \begin{exe}
    \ex Das ist ein Kneck.
    \pause
    \vspace{\baselineskip}
  \ex Jede Farbe ist ein Kurzwellenradio.
  \ex Der dichte Tank leckt.
\end{exe}
    \vspace{\baselineskip}
  \pause

  \Large\begin{block}{Kompositionalität}
    Die Bedeutung komplexer sprachlicher Ausdrücke ergibt sich aus der Bedeutung ihrer Teile und der Art ihrer grammatischen Kombination. 
    Diese Eigenschaft von Sprache nennt man Kompositionalität.
  \end{block}
\end{frame}

\begin{frame}
  {Grammatik als System und Grammatikalität}
  \pause

  \Large\begin{block}{Grammatik}
    Eine Grammatik ist ein \alert{System von Regularitäten}, nach denen aus einfachen Einheiten komplexe Einheiten einer Sprache gebildet werden.
  \end{block}
  \vspace{\baselineskip}

  \pause

  \begin{block}{Grammatikalität}
    Jede von einer bestimmten Grammatik beschriebene Symbolfolge ist \alert{grammatisch} relativ zu dieser Grammatik, alle anderen sind \alert{ungrammatisch}.
  \end{block}
\end{frame}

\begin{frame}
  {(Un)grammatisch ist nicht gleich (in)akzeptabel}
  \pause
  \begin{exe}
    \ex\begin{xlist}
      \ex Bäume wachsen werden hier so schnell nicht wieder.
      \pause
      \ex Touristen übernachten sollen dort schon im nächsten Sommer.
      \pause
      \ex Schweine sterben müssen hier nicht.
      \pause
      \ex Der letzte Zug vorbeigekommen ist hier 1957.
      \pause
      \ex Das Telefon geklingelt hat hier schon lange nicht mehr.
      \pause
      \ex Häuser gestanden haben hier schon immer.
      \pause
      \ex Ein Abstiegskandidat gewinnen konnte hier noch kein einziges Mal.
      \pause
      \ex Ein Außenseiter gewonnen hat hier erst letzte Woche.
      \pause
      \ex Die Heimmannschaft zu gewinnen scheint dort fast jedes Mal.
      \pause
      \ex Ein Außenseiter gewonnen zu haben scheint hier noch nie.
      \pause
      \ex Ein Außenseiter zu gewinnen versucht hat dort schon oft.
      \pause
      \ex Einige Außenseiter gewonnen haben dort schon im Laufe der Jahre.
    \end{xlist}
  \end{exe}
\end{frame}

\begin{frame}
  {Kern und Peripherie}
  \pause
\begin{exe}
  \ex\label{ex:kernundperipherie022}
    \begin{xlist}
      \ex \alert{Baum, Haus, Matte, Döner, Angst, Öl, Kutsche, \ldots}
      \ex \rot{System, Kapuze, Bovist, Schlamassel, Marmelade, Melodie, \ldots}
    \end{xlist}
    \pause
    \ex
    \begin{xlist}
      \ex \alert{geht, läuft, lacht, schwimmt, liest, \ldots}
      \ex \rot{kann, muss, will, darf, soll, mag}
    \end{xlist}
    \pause
    \ex
    \begin{xlist}
      \ex \alert{des Hundes, des Geistes, des Tisches, des Fußes, \ldots}
      \ex \rot{des Schweden, des Bären, des Prokuristen, des Phantasten, \ldots}
    \end{xlist}
  \end{exe}
  \pause
  \vspace{\baselineskip}
  \Large
  \centering
  \alert{Hohe Typenhäufigkeit} vs.\ \rot{niedrige Typenhäufigkeit}.  
\end{frame}

\begin{frame}
  {Zwei verschiedene Häufigkeiten}
  \pause
  \Large\begin{block}{Typenhäufigkeit}
    Wie viele \alert{verschiedene} Realisierungen (=~Typen)\\
    einer Sorte linguistischer Einheiten gibt es?
  \end{block}

  \pause
  \vspace{\baselineskip}
  
  \begin{block}{Tokenhäufigkeit}
    Wie häufig sind die \alert{ggf.\ identischen} Realisierungen\\
    (=~Tokens) einer Sorte linguistischer Einheiten?
  \end{block}
\end{frame}

\begin{frame}
  {Regel vs.\ Regularität bzw.\ Generalisierung}
  \pause
  \begin{exe}
    \ex
    \begin{xlist}
      \ex{Relativsätze und eingebettete \textit{w}-Sätze werden nicht\\
    durch Komplementierer eingeleitet.}
      \pause
      \ex{\textit{fragen} ist ein schwaches Verb.}
      \pause
      \ex{\textit{zurückschrecken} bildet das Perfekt mit dem Hilfsverb \textit{sein}.}
      \pause
      \ex{Im Aussagesatz steht vor dem finiten Verb genau ein Satzglied.}
      \pause
      \ex{In Kausalsätzen mit \textit{weil} steht das finite Verb an letzter Stelle.}
    \end{xlist}
  \end{exe}
\end{frame}


\begin{frame}
  {Normkorm? Regularitätenkonform?}
  \pause
  \begin{exe}
    \ex
    \begin{xlist}
      \ex Dann sieht man auf der ersten Seite \alert{wann, wo und wer} \rot{dass} kommt.
      \pause
      \ex Er \rot{frägt} nach der Uhrzeit.
      \pause
      \ex Man \rot{habe} zu jener Zeit nicht vor Morden \alert{zurückgeschreckt}.
      \pause
      \ex \rot{Der Universität} \alert{zum Jubiläum} gratulierte auch Bundesminister Dorothee Wilms, die in den fünfziger Jahren in Köln studiert hatte.
      \pause
      \ex Das ist Rindenmulch, \alert{weil} hier \rot{kommt} noch ein Weg.
    \end{xlist}
  \end{exe}
\end{frame}


\begin{frame}
  {Regel und Regularität}
  \pause
  \begin{block}{Regularität}
    Eine grammatische Regularität innerhalb eines Sprachsystems liegt dann vor, wenn sich Klassen von Symbolen unter vergleichbaren Bedingungen gleich (und damit vorhersagbar) verhalten.
  \end{block}

  \pause
  \vspace{0.5\baselineskip}

  \begin{block}{Regel}
    Eine grammatische Regel ist die Beschreibung einer Regularität, die in einem normativen Kontext geäußert wird.
  \end{block}

  \pause
  \vspace{0.5\baselineskip}
  
  \begin{block}{Generalisierung}
    Eine grammatische Generalisierung ist eine durch Beobachtung zustandegekommene Beschreibung einer Regularität.
  \end{block}
\end{frame}

\begin{frame}
  {Regel vs.\ Regularität bzw.\ Generalisierung}
  Was ist dann der Status dieser Feststellungen?\\
  \Zeile
  \begin{exe}
    \ex
    \begin{xlist}
      \ex{Relativsätze und eingebettete \textit{w}-Sätze werden nicht\\
    durch Komplementierer eingeleitet.}
      \ex{\textit{fragen} ist ein schwaches Verb.}
      \ex{\textit{zurückschrecken} bildet das Perfekt mit dem Hilfsverb \textit{sein}.}
      \ex{Im Aussagesatz steht vor dem finiten Verb genau ein Satzglied.}
      \ex{In Kausalsätzen mit \textit{weil} steht das finite Verb an letzter Stelle.}
    \end{xlist}
  \end{exe}
\end{frame}



\begin{frame}
  {Norm ist Beschreibung}
  \pause
  \begin{itemize}[<+->]
    \item Norm als Grundkonsens
    \item Sprache und Norm im Wandel
    \item Norm und Situation (Register, Stil, \dots)
    \item Variation in der Norm
      \vspace{\baselineskip}
    \item \alert{Wichtigkeit der Norm, insbesondere im schulischen Deutschunterricht}
  \end{itemize}
\end{frame}


